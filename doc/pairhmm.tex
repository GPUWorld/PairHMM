\documentclass[11pt, oneside]{article}

\title{Enabling high throughput haplotype analysis through hardware acceleration} 
\author{Carneiro MO, Biagioli E, Pratt-Szeliga P, Thibault S, Vacek G, Nehab D}

\begin{document} \maketitle
	
	\begin{abstract} 
	
	Small variants such as single nucleotide polymorphisms (SNPs) or single base
	pair insertions or deletions (indel) are not the most difficult task for
	variant calling algorithms. Latest call sets from the 1000 Genomes project
	have shown upwards of 99\% sensitivity on those types of variants (cite
	1kg). Indels extending up to a fraction of the sequencing read lengths are
	the current challenge of variant detection, and with ever-extending read
	lengths, the expectation of identifying larger indels increases at the same
	rate. Tools like the GATK's Haplotype Caller (cite GATK/HC) rely on a pair
	Hidden Markov Model (HMM - cite book) to score the probability of the pairwise alignment
	between a read and a candidate haplotype which is responsible for over 50\%
	of the runtime. Here we present a framework that enables the analysis of
	whole genome sequencing data with such algorithms through a high throughput
	implementation of the HMM using two different types of hardware
	accelerations: Graphics Processing Unit (GPU) and Field Programmable Gate
	Array (FPGA).
	
	\end{abstract}

	\section{Introduction} 

	Calling larger indels and short rearrangements (up to 50 bases
	long) can only be achieved through local denovo assembly of a
	reference haplotype based on the read data (cite haplotype caller?).
	For this computation we rely on fast heuristic methods that generate
	pcandidate haplotypes without trying all possibilities (cite debruijn
	graphs). Once we have a list of candidate haplotypes, we
	have to determine which of these haplotypes is the most likely
	haplotype to explain the read data. For this task we use a paired
	hidden markov model (PairHMM) that calculates the probability of
	all alignments of a read to a candidate haplotype, repeating the
	process for every candidate haplotype against every read and
	tallying the results. The haplotype with the highest probability
	across all reads is the chosen one. This PairHMM is responsible for
	more than 50\% of the runtime of the Haplotype Caller (denovo
	assembly variant caller available in the Genome Analysis
	Toolkit), and gets worse for longer reads, due to it's $N^2$ implementation. Here we show how
	we can harness the parallelism of graphics processing units (GPU)
	and field-programmable gate arrays (FPGA) to reduce the runtime by ~300
	fold. This performance improvement not only accelerates the calling process,
	but also enables projects that need to call tens of thousands of samples
	which would otherwise be impossible.  
	

	\section{Methods} 
	
	\subsection{Mathematical Description of the Problem} 
	
	Full description of the HMM in it's final version.  (Mauricio) 
	
	\subsection{CPU implementation} 
	
	Describe the GATK's implementation of the PairHMM. (Mauricio) 
	
	\subsection{FPGA Implementation} 
	
	We also investigated optimizing the PairHMM algorithm on Convey Computer’s
	Hybrid-Core (HC) architecture, which previously has proven useful for a
	range of bioinformatics applications including BWA. HC servers combine a
	traditional x86 environment with a high-performance coprocessor based on
	Field Programmable Gate Arrays (FPGAs).  FPGAs are reprogrammable chips that
	can be loaded with instructions for the critical, computationally intensive
	blocks of an algorithm, achieving significant speedups relative to
	general-purpose CPUs. Their ability to be dynamically reprogrammed at run
	time allows them to deliver such optimized performance to many different
	applications.   Coprocessor instruction sets for the HC, or personalities,
	are embedded in applications like the GATK to be automatically used when the
	application runs if a coprocessor is present.  The executable also includes
	the standard x86 instruction set, which is used when running on a
	traditional server where no coprocessor is present.  The paired Hidden
	Markov Model (PairHMM) personality was developed for the HC in Impulse C,
	a personality development productivity tool based on the C language.  The
	PairHMM personality performs comparisons for every base in the read sequence
	in parallel by pipelining the instructions as a systolic array with 162
	elements, which can compare that many bases with a candidate haplotype
	simultaneously.  For the calculation of these direct probabilities, for each
	comparison we calculate the three Viterbi values
	(match, insertion and deletion probabilities) in parallel as well.  The pipeline can be
	envisioned as a wavefront, where the haplotype streams across the sequence
	of read bases to fill out the Viterbi matrix of all comparisons.  The
	processing element emits the full set of Viterbi values for each cell on
	every instruction.  That is, in the first instruction it computes $V_{1,1}$; the
	second computes both $V_{1,2}$ and $V_{2,1}$; the third computes all of $V_{1,3}$, $V_{2,2}$,
	and $V_{3,1}$; etc., with the $j^{th}$ instruction computing $V_{i,j−i+1}$
	for all rows $i$.
	
	For such systolic arrays, top performance is achieved when the pipe remains
	full as many cycles as possible.  In this case, performance is better for
	longer haplotypes, and when reads are as close to 162 bases in length.  The
	read limit is a result of the available resources on the FPGA chips of HC-1
	and HC-2 servers.  HC-2ex servers have a larger coprocessor and would yield
	about a three-fold performance boost over the HC-2.  The additional
	resources could be used for longer read lengths or for more simultaneous
	operations in parallel of the same read lengths with no fundamental change
	in design.  

	
	\subsection{CUDA Implementation} 
	
	Modern GPUs are massively parallel computing devices composed of multiple
	independent multiprocessors, each capable of executing the same operation on
	multiple data elements simultaneously. The architecture is designed for high
	throughput, rather than for low latency.  For best performance, the
	computation must be divided according to this two-level hierarchy: a number
	of independent jobs large enough to keep all multiprocessors busy
	(inter-block parallelism), each job to be executed by collaborating cores
	that perform computations in tandem (intra-block parallelism).

	To this end, we enqueue as many pairs as will fit into the GPU memory before
	launching a persistent kernel.  Each computational block loops obtaining
	pairs from the queue until it is empty. This setup supplies the necessary
	inter-block parallelism.  Due to the restricted amount of shared-memory
	available, and to enable a larger number of resident blocks, we break the
	traversal of matrices~$X$, $Y$ and~$M$ into horizontal bands. Finally, since
	diagonal entries are independent from one another, we evaluate diagonal
	elements in parallel thereby exposing the required intra-block parallelism. 


	\subsection{Rootbeer Implementation} 
	
	Describe the Rootbeer implementation of the PairHMM. Why is it different from CUDA. (Phil) 
	
	\subsection{Experimental data and methods} 
	
	1000G data reference, talk about the GATK pipeline and how we made these
	calls and prepare the test data sets. (Mauricio) 
	
	\section{Results}

	Comparison of the different implementations running on NA12878 data.  How
	does FPGA and CUDA differ both implementationwise and performance wise?
	What is the difference from the user's perspective, what do they have to do
	differently?  
	
	Performance of the FPGA implementation was measured on an HC-1 with 128GB of
	host memory and 64GB of coprocessor scatter-gather memory with a 2-way
	stripe of 1TB disks.  The host has an L5408 quad core processor running at
	2.13GHz.  The largest data set with 14.1M test cases ran in 770 seconds on
	the HC-1 as compared with about 10800 seconds on a 12-core server with the
	final optimized software (and about 68200 seconds with the original code).
	The PairHMM personality on the HC was implemented with single precision
	instructions.  All results agree with the software to 10**-5 accuracy,
	showing that this is sufficient and double precision instructions are not
	required

	\subsection{Experimental results on 1000G data} 
	
	What did we find? New SNPs? New Indels?  What were the benefits of calling
	this data together using the haplotype caller?  Why is this useful to other
	medically relevant projects around the world?  
	
	\subsection{Availability of hardware acceleration as a service} 

	Cloud computing offers the ability to flexibly use computing on demand when
	it is needed, and thus to rent it on a per project basis rather than
	purchase servers up front.  It can also provide an opportunity to shift to
	the most appropriate HPC architecture when various applications in a
	workflow would benefit from different resources (coprocessors, high memory,
	etc).  Amazon Web Services (AWS), probably the best known cloud
	service provider, offers both standard x86 servers as well as Cluster GPU
	instances [aws.amazon.com].  Nimbix offers an Accelerated Compute Cloud
	(NACC), with a bioinformatics portal to the widest availability of
	accelerated bioinformatics computing, including Convey HC servers as well as
	standard and GPUs servers [www.nimbix.net].

\end{document}

